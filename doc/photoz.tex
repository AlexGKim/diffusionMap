\documentclass[preprint]{aastex}
\usepackage{amsmath,amssymb}
\usepackage{mathrsfs}
\bibliographystyle{plain}


\begin{document}

\title{Photometric Redshift Subsamples}
\author{Alex Kim}

\section{Introduction}

The determination of galaxy redshifts from broad-band photometry (photometric redshifts) can provide redshift estimates
for the large numbers of galaxies observed with multi-band imaging.  Most  galaxies do not have a
spectroscopic redshift, so the use of photometric redshifts enables a bread range of science.  
Given its importance, a diverse range of photometric-redshift algorithms are now available.

Without fine spectral resolution, photometric redshift estimates are less precise and easily less accurate than redshifts
determined from spectral features.  All algorithms to date exhibit non-Gaussian tails, referred to as catastrophic errors,
in their distribution of the difference between photometric and spectroscopic redshifts (redshift error).
These catastrophic errors limit
the statistical power of a single galaxy redshift, and the accuracy of the ensemble of photometric redshifts.

Although catastrophic errors plague the full ensemble of galaxies, there may be a subset of  galaxies identifiable through
photometry that is free of catastrophic errors and/or has small dispersion in redshift error.  This article suggests
criteria for defining such a subset, given a set of galaxies with
photometric and spectroscopic redshifts.  One criterion is that a galaxy must be similar to other galaxies in the sample,
under the assumption that ``singleton'' galaxies are less likely to have a robust photometric redshift.
The second criterion is that a galaxy must be dissimilar to galaxies that have catastrophic error, making the assumption
that such errors do not occur uniformly for all galaxies but preferentially in identifiable subclasses.
The quantitative definition of ``similar'' and ``dissimilar'' is based on distances between galaxies; a galaxy with small
distances to many other galaxies is considered similar, whereas a galaxy with large distances to the catastrophic outliers
is considered dissimilar.

Two distances are considered in this article.  The first is the Euclidian distance between points in color-magnitude
space.  The second is the distance of a particular choice for the non-linear reparameterization of color-magnitude space:
the Diffusion Map distance represents the connectivity between two points through a diffusion process on a graph.

\end{document}